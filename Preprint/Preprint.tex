\documentclass[]{elsarticle} %review=doublespace preprint=single 5p=2 column
%%% Begin My package additions %%%%%%%%%%%%%%%%%%%
\usepackage[hyphens]{url}

  \journal{SportR\(\chi\)iv} % Sets Journal name


\usepackage{lineno} % add
\providecommand{\tightlist}{%
  \setlength{\itemsep}{0pt}\setlength{\parskip}{0pt}}

\usepackage{graphicx}
\usepackage{booktabs} % book-quality tables
%%%%%%%%%%%%%%%% end my additions to header

\usepackage[T1]{fontenc}
\usepackage{lmodern}
\usepackage{amssymb,amsmath}
\usepackage{ifxetex,ifluatex}
\usepackage{fixltx2e} % provides \textsubscript
% use upquote if available, for straight quotes in verbatim environments
\IfFileExists{upquote.sty}{\usepackage{upquote}}{}
\ifnum 0\ifxetex 1\fi\ifluatex 1\fi=0 % if pdftex
  \usepackage[utf8]{inputenc}
\else % if luatex or xelatex
  \usepackage{fontspec}
  \ifxetex
    \usepackage{xltxtra,xunicode}
  \fi
  \defaultfontfeatures{Mapping=tex-text,Scale=MatchLowercase}
  \newcommand{\euro}{€}
\fi
% use microtype if available
\IfFileExists{microtype.sty}{\usepackage{microtype}}{}
\bibliographystyle{elsarticle-harv}
\ifxetex
  \usepackage[setpagesize=false, % page size defined by xetex
              unicode=false, % unicode breaks when used with xetex
              xetex]{hyperref}
\else
  \usepackage[unicode=true]{hyperref}
\fi
\hypersetup{breaklinks=true,
            bookmarks=true,
            pdfauthor={},
            pdftitle={Short Paper},
            colorlinks=false,
            urlcolor=blue,
            linkcolor=magenta,
            pdfborder={0 0 0}}
\urlstyle{same}  % don't use monospace font for urls

\setcounter{secnumdepth}{0}
% Pandoc toggle for numbering sections (defaults to be off)
\setcounter{secnumdepth}{0}

\newlength{\cslhangindent}
\setlength{\cslhangindent}{1.5em}
\newenvironment{cslreferences}%
  {\setlength{\parindent}{0pt}%
  \everypar{\setlength{\hangindent}{\cslhangindent}}\ignorespaces}%
  {\par}

% Pandoc header
\usepackage{booktabs}
\usepackage{longtable}
\usepackage{array}
\usepackage{multirow}
\usepackage{wrapfig}
\usepackage{float}
\usepackage{colortbl}
\usepackage{pdflscape}
\usepackage{tabu}
\usepackage{threeparttable}
\usepackage{threeparttablex}
\usepackage[normalem]{ulem}
\usepackage{makecell}
\usepackage{xcolor}



\begin{document}
\begin{frontmatter}

  \title{Short Paper}
    \author[Centre for Sport Research]{Aaron S. Fox\corref{1}}
  
    \author[Centre for Sport Research]{Tanisha Bardzinski}
  
    \author[Centre for Sport Research]{Lyndell Bruce}
  
      \address[Centre for Sport Research]{Centre for Sport Research,
School of Exercise and Nutrition Sciences, Deakin University, Geelong,
Australia}
      \cortext[1]{Corresponding Author: aaron.f@deakin.edu.au}
  
  \begin{abstract}
  Insert abstract\ldots{}
  \end{abstract}
  
 \end{frontmatter}

\hypertarget{todo}{%
\section{TODO:}\label{todo}}

\begin{itemize}
\tightlist
\item
  Check consistency of Super Shot vs.~Power 5 period (use Power 5
  period)
\item
  Look to ESSA abstract for some sample points and language
\end{itemize}

\hypertarget{introduction}{%
\section{Introduction}\label{introduction}}

Netball is a court-based team sport played predominantly among
Commonwealth nations, and has one of the highest participation rates for
team sports in Australia ({\textbf{???}}). As in many court-based team
sports, the goal of netball is to score more than the opposition.
Netball is, however, unique in that goals may only be scored by two
players on each team from within the `shooting circle' (i.e.~a half
circle around the goal with a 4.9m radius) at their end of the court
({\textbf{???}}). Traditionally, goals scored from within this circle
result in one `goal' or `point' for the team ({\textbf{???}}). In the
2020 season, Australia's national elite-level league (i.e.~Suncorp Super
Netball) made the decision to introduce the `Super Shot'
({\textbf{???}}). The Super Shot period provided teams an opportunity to
gain one- versus two-points for successful shots made from the `inner'
(i.e.~0m-3.0m) versus `outer' (i.e.~3.0m-4.9m), respectively, within the
final five minutes of each quarter (i.e.~the Power 5 period)
({\textbf{???}}). The league confirmed that the Super Shot rule is
continuing through the 2021 season ({\textbf{???}}).

Our analysis prior to the 2020 season ({\textbf{???}}) suggested that
the added value of the Super Shot (i.e.~two-points) aligned well with
the elevated risk of shooting from long range, and that teams may have
been able to maximise their scoring by taking a high proportion of Super
Shots. These findings were, however, based on shooting statistics from
past seasons where the Super Shot rule was not in effect -- and further
investigation of leagues where a `two-point rule' was in place
(i.e.~international Fast5) resulted in a much higher risk of missing
long-range shots ({\textbf{???}}). We hypothesised that the elevated
risk of missing long-range shots with a `two-point rule' in place stems
from situational factors, whereby defensive strategies were likely
altered to place a heavier emphasis on defending long-range shots
({\textbf{???}}). Data from the first full season with the Super Shot in
place provides an opportunity to re-evaluate the risk-reward value of
taking Super Shots with more valid shooting statistics. Further, these
data can provide a better foundation for simulating Super Shot periods
as a means to identify optimal shooting strategies. In the present study
we firstly re-visited the question of whether the weighting of a 2:1
value is appropriate based on the relative risk of missing a shot from
the outer versus inner circle during the Power 5 period using data from
the 2020 season. Second -- we ran simulations of the Power 5 period for
each team, driven by shooting statistics from the 2020 season, in an
attempt to identify optimal team-specific shooting strategies for the
proportion of Super Shots to take. Third -- we ran simulations of teams
competing against one another during a Power 5 period to determine how
varying the proportion of Super Shots relative to total shots could
impact scoring margin.

\hypertarget{methods}{%
\section{Methods}\label{methods}}

\hypertarget{participants}{%
\subsection{Participants}\label{participants}}

Participants for this study included all players across the eight teams
from the 2020 season of the Australian national netball league
(i.e.~Suncorp Super Netball). Our study included publicly available,
pre-existing data held on the Suncorp Super Netball match centre
(\textbf{\emph{TODO: add link}}). An exemption from ethics review (and
subsequent waiver of individual consent) was granted by the Deakin
University Human Research Ethics Committee (\textbf{\emph{TODO: add
details}}).

\hypertarget{data-collection}{%
\subsection{Data Collection}\label{data-collection}}

We used the \{SuperNetballR\}\textbf{\emph{TODO: citation}} package to
extract match data from all regular season games during the 2020 Super
Netball Season via the Champion Data (official provider of competition
statistics) match centre. Within the match centre data -- all shots are
labelled with identifiers that place them in the inner or outer circle,
along with whether they were made or missed. Combined with the timestamp
of these events within quarters, we extracted team-specific shooting
statistics for: (i) the total number of shots taken; (ii) the number of
shots taken from the inner and outer circle; and (iii) the number of
made and missed shots from the inner and outer circle from each Power 5
period across the season.

\hypertarget{data-analysis}{%
\subsection{Data Analysis}\label{data-analysis}}

Our study required estimating the probability of making versus missing
shots from the inner versus outer circle across the different teams. We
achieved this by defining a beta distribution in a probability density
function for the different circle zones, specified by:

\[f(x,a,b) = \frac{\Gamma(a+b)x^{a-1}(1-x)^{b-1}}{\Gamma(a)\Gamma(b)} \]
where \(a\) and \(b\) represent the number of missed and made shots
within a circle zone, respectively; \(x\) is the probability of \(a\)
relative to \(b\); and \(\Gamma\) is the gamma function
({\textbf{???}}). Probability density functions were created for made
versus missed shots in the inner and outer circles for each team, as
well as all teams combined, to be used in subsequent analyses.

To examine the relative value of the 2:1 point ratio, we replicated the
approach from our previous work ({\textbf{???}}), but this time with
data from the 2020 season. Specifically, we compared the average
relative odds (± 95\% confidence intervals {[}CI{]}) of missing from the
outer versus inner circle during the Power 5 period. This was achieved
by dividing randomly sampled values (\emph{n} = 100,000) from the
probability density functions of the outer by those from the inner
circle at each sample iteration. This analysis was run using shooting
statistics from the entire league, as well as individual teams, to give
overall and team-specific risk-reward values for attempting Super Shots.
We also applied this analysis to opposition shooting statistics against
each team, providing a risk-reward value for attempting Super Shots
against an opponent. Theoretically, the relative odds of missing from
the outer to inner circle should match the ratio of points awarded
(i.e.~2:1) for the Super Shot to represent `good value.'

Next -- we ran a series of simulations (\emph{n} = 1,000 each) of the
Power 5 period for each team, altering the proportion of Super Shots
taken from 0\% to 100\% at 10\% increments. We used the previously
calculated probabilities of making versus missing shots from within and
outside the outer circle during the Power 5 period from each team to
estimate the number of points the team may score during this period.
Across each individual simulation, the total number of shots the team
would take and the proportion of these that were Super Shots was
determined. We calculated the mean and standard deviation for the number
of shots a team would expect during a Power 5 period based on overall
season statistics --- and the total number of shots for a team in an
individual simulation was randomly sampled from a truncated normal
distribution between the lower and upper 95\% CI limits of the
mean/standard deviation. The number of standard versus Super Shots being
taken within the simulation was then determined based on the current
Super Shot proportion (i.e.~from 0\% to 100\%) being examined. The
success (i.e.~make vs.~miss) of each individual standard or Super Shot
within the simulation was then determined by generating a random value
between 0 and 1, alongside a value sampled from the teams relevant
probability density function of making a shot from the relevant location
(i.e.~inner or outer circle). If the value sampled from the probability
density function was greater versus lower than the random value --- the
shot was considered successful versus unsuccessful, respectively. After
all individual shots were simulated, the total team score was summed
given the value of the made standard and Super shots. We examined the
scores achieved, and calculated the relative frequency of when the
minimum and maximum score was achieved for each team under the different
Super Shot proportions.

A similar approach was taken in simulating teams competing against one
another during Power 5 periods. A series of simulations (\emph{n} =
1,000) of Power 5 periods were ran between all combinations of teams. We
once again used the probabilities of making versus missing shots from
within and outside the outer circle during the Power 5 period from each
team to estimate scoring. To determine the number of shots each team
received in a simulation, we created normal distribution based on the
mean and standard deviation of the total shots from both teams and the
proportion of these shots taken by each team in Power 5 periods across
the season. Values were sampled from these normal distributions within
each individual simulation to determine the number of shots each team
were allocated. As part of this approach we ensured that appropriate
balance was achieved for the shots allocated between teams by allocating
the first team the relevant proportion of shots, and then the opposing
team the remaining shots (i.e.~if one team received a high proportion of
the total shots, the opposing team received a low proportion of the
total shots). Each series of 1,000 simulations was repeatedly ran
between all combinations of teams while altering the proportion of Super
versus standard shots. For brevity in these simulations, the proportion
of Super Shots taken by each team was altered from 0\% to 100\% at 25\%
increments - with every possible combination between teams simualted.
Shot success was determined in the same manner as previously outlined
(i.e.~random number generator vs.~value sampled from the teams shot
success probability distribution). After the shots from both teams were
simulated, each teams score was summed given the value of made standard
and Super shots and the subsequent margin determined. The mean and 95\%
CIs for margins between each team across the various Super Shot
proportion combinations simulated were then calculated.

\textbf{\emph{TODO: any details about comparing shot numbers/shooting
percentages?}}

\begin{itemize}
\tightlist
\item
  Include some probability calculations of winning vs.~losing the 5
  minute period (i.e.~X team's relative probability of winning
  vs.~losing the super shot period under different proportions; or maybe
  even between proportions?)
\end{itemize}

\hypertarget{results}{%
\section{Results}\label{results}}

The relative combined odds (± 95\% CIs) from all teams of missing from
the outer versus the inner circle across the entire match were 4.19
{[}3.88, 4.52{]}, and 4.04 {[}3.5, 4.6{]} versus 4.68 {[}4.07, 5.35{]}
in the standard and Super Shot periods, respectively. The relative odds
of missing from the outer versus inner circle across the individual
teams were relatively similar, with the exception of the Fever having
higher odds than the majority of teams across the entire match period
(see Figure \ref{fig:relativeOddsFig}). The relative odds (± 95\% CIs)
of missing from the outer versus inner circle during the Super Shot
Period were greater than 2:1 across all teams (see Figure
\ref{fig:relativeOddsFig}). No team appeared more or less effective in
elevating the risk of missing from the outer versus inner circle, with
mostly similar odds observed across all teams opponents in the various
periods of the quarter (see Figure \ref{fig:relativeOddsDefFig}).

\begin{figure}

{\centering \includegraphics[width=1\linewidth]{C:/+GitRepos+/super-shot-sims/Results/relativeOdds/figures/RelativeOdds_OuterInner_AllTeams} 

}

\caption{Relative odds (mean $\pm$ 95$\%$ confidence intervals) of teams missing from the outer versus inner circle across the entire match, and during the standard and Super Shot scoring periods.}\label{fig:relativeOddsFig}
\end{figure}

\begin{figure}

{\centering \includegraphics[width=1\linewidth]{C:/+GitRepos+/super-shot-sims/Results/relativeOdds/figures/RelativeOddsDef_OuterInner_AllTeams} 

}

\caption{Relative odds (mean $\pm$ 95$\%$ confidence intervals) of opposition teams missing from the outer versus inner circle across the entire match, and during the standard and Super Shot scoring periods.}\label{fig:relativeOddsDefFig}
\end{figure}

A similar pattern was observed across all teams when simulating Power 5
periods with varying Super Shot proportions. Specifically, more
consistent but lower scores were generated using a lower proportion of
Super Shots (i.e.~\textless{} 30\%) versus inconsistent high and low
scores when using a higher proportion of Super Shots
(i.e.~\textgreater{} 80\%) (see Figure \ref{fig:heatMapFig} and
\textbf{\emph{TODO: supplementary figures of individual team box
plots}}). Effectively, both the maximum and minimum scores achievable
increased and decreased, respectively, as Super Shot proportion
increased (i.e.~the highest and lowest achievable scores came from
simulations using a high proportion of Super Shots).

\begin{landscape}
\begin{figure}

{\centering \includegraphics[width=1\linewidth]{C:/+GitRepos+/super-shot-sims/Results/standardSims/figures/SuperShotSimulations_HeatMap_AllTeams_Abs} 

}

\caption{Distribution of simulated score outputs for each team using different Super Shot proportions. Black outline represents each teams actual proportion of Super Shots taken during Power 5 periods from 2020 season statistics.}\label{fig:heatMapFig}
\end{figure}
\end{landscape}

\ldots standard sim max and min results\ldots{}

\begin{table}

\caption{\label{tab:standardSimMaxSummaryTable}TODO Add caption for max results...}
\centering
\resizebox{\linewidth}{!}{
\begin{tabular}[t]{lrrrrrrrrr}
\toprule
  & 10\%-20\% & 20\%-30\% & 30\%-40\% & 40\%-50\% & 50\%-60\% & 60\%-70\% & 70\%-80\% & 80\%-90\% & 90\%-100\%\\
\midrule
\cellcolor{gray!6}{Fever} & \cellcolor{gray!6}{9.9} & \cellcolor{gray!6}{6.7} & \cellcolor{gray!6}{1.9} & \cellcolor{gray!6}{9.7} & \cellcolor{gray!6}{21.9} & \cellcolor{gray!6}{2.8} & \cellcolor{gray!6}{9.7} & \cellcolor{gray!6}{23.1} & \cellcolor{gray!6}{12.6}\\
Firebirds & 6.5 & 0.0 & 22.1 & 6.3 & 5.4 & 14.7 & 12.6 & 15.7 & 14.6\\
\cellcolor{gray!6}{GIANTS} & \cellcolor{gray!6}{2.8} & \cellcolor{gray!6}{0.0} & \cellcolor{gray!6}{13.7} & \cellcolor{gray!6}{8.5} & \cellcolor{gray!6}{0.4} & \cellcolor{gray!6}{24.2} & \cellcolor{gray!6}{2.4} & \cellcolor{gray!6}{29.5} & \cellcolor{gray!6}{18.5}\\
Lightning & 3.0 & 4.5 & 3.2 & 8.4 & 16.6 & 7.0 & 11.5 & 28.5 & 16.8\\
\cellcolor{gray!6}{Magpies} & \cellcolor{gray!6}{0.5} & \cellcolor{gray!6}{5.8} & \cellcolor{gray!6}{3.8} & \cellcolor{gray!6}{30.0} & \cellcolor{gray!6}{1.3} & \cellcolor{gray!6}{0.0} & \cellcolor{gray!6}{26.4} & \cellcolor{gray!6}{0.0} & \cellcolor{gray!6}{31.7}\\
\addlinespace
Swifts & 4.3 & 5.5 & 4.3 & 9.6 & 15.6 & 7.7 & 8.1 & 28.8 & 15.7\\
\cellcolor{gray!6}{Thunderbirds} & \cellcolor{gray!6}{4.0} & \cellcolor{gray!6}{0.0} & \cellcolor{gray!6}{28.0} & \cellcolor{gray!6}{1.8} & \cellcolor{gray!6}{9.6} & \cellcolor{gray!6}{4.0} & \cellcolor{gray!6}{29.0} & \cellcolor{gray!6}{3.5} & \cellcolor{gray!6}{18.7}\\
Vixens & 2.3 & 0.1 & 11.3 & 8.2 & 0.3 & 23.8 & 0.1 & 34.7 & 19.1\\
\bottomrule
\end{tabular}}
\end{table}

\begin{table}

\caption{\label{tab:standardSimMinSummaryTable}TODO Add caption for min results...}
\centering
\resizebox{\linewidth}{!}{
\begin{tabular}[t]{lrrrrrrrrr}
\toprule
  & 10\%-20\% & 20\%-30\% & 30\%-40\% & 40\%-50\% & 50\%-60\% & 60\%-70\% & 70\%-80\% & 80\%-90\% & 90\%-100\%\\
\midrule
\cellcolor{gray!6}{Fever} & \cellcolor{gray!6}{2.2} & \cellcolor{gray!6}{2.4} & \cellcolor{gray!6}{1.3} & \cellcolor{gray!6}{6.3} & \cellcolor{gray!6}{16.4} & \cellcolor{gray!6}{2.1} & \cellcolor{gray!6}{12.4} & \cellcolor{gray!6}{35.9} & \cellcolor{gray!6}{20.9}\\
Firebirds & 6.5 & 0.0 & 16.5 & 6.9 & 4.6 & 12.7 & 17.7 & 14.8 & 17.4\\
\cellcolor{gray!6}{GIANTS} & \cellcolor{gray!6}{10.8} & \cellcolor{gray!6}{0.0} & \cellcolor{gray!6}{15.5} & \cellcolor{gray!6}{9.2} & \cellcolor{gray!6}{0.5} & \cellcolor{gray!6}{21.5} & \cellcolor{gray!6}{2.2} & \cellcolor{gray!6}{22.4} & \cellcolor{gray!6}{13.4}\\
Lightning & 8.9 & 4.7 & 2.9 & 9.1 & 16.5 & 4.1 & 9.8 & 27.1 & 14.6\\
\cellcolor{gray!6}{Magpies} & \cellcolor{gray!6}{0.8} & \cellcolor{gray!6}{9.1} & \cellcolor{gray!6}{2.6} & \cellcolor{gray!6}{23.0} & \cellcolor{gray!6}{1.4} & \cellcolor{gray!6}{0.0} & \cellcolor{gray!6}{29.8} & \cellcolor{gray!6}{0.0} & \cellcolor{gray!6}{31.1}\\
\addlinespace
Swifts & 8.4 & 5.1 & 4.2 & 8.0 & 13.7 & 6.1 & 9.1 & 27.9 & 15.1\\
\cellcolor{gray!6}{Thunderbirds} & \cellcolor{gray!6}{3.7} & \cellcolor{gray!6}{0.0} & \cellcolor{gray!6}{18.4} & \cellcolor{gray!6}{1.5} & \cellcolor{gray!6}{7.8} & \cellcolor{gray!6}{3.9} & \cellcolor{gray!6}{39.1} & \cellcolor{gray!6}{6.0} & \cellcolor{gray!6}{17.5}\\
Vixens & 11.2 & 0.2 & 18.5 & 9.8 & 0.1 & 21.7 & 0.2 & 22.9 & 12.9\\
\bottomrule
\end{tabular}}
\end{table}

\ldots{}

\begin{landscape}\begin{table}

\caption{\label{tab:marginTable}TODO Add caption}
\centering
\resizebox{\linewidth}{!}{
\begin{tabular}[t]{lllllllll}
\toprule
X & Fever & Firebirds & GIANTS & Lightning & Magpies & Swifts & Thunderbirds & Vixens\\
\midrule
\cellcolor{gray!6}{Team 0\% / Opp. 0\%} & \cellcolor{gray!6}{0.25 [0.06, 0.45]} & \cellcolor{gray!6}{-0.51 [-0.70, -0.32]} & \cellcolor{gray!6}{-0.10 [-0.29, 0.09]} & \cellcolor{gray!6}{-0.01 [-0.20, 0.19]} & \cellcolor{gray!6}{0.13 [-0.07, 0.32]} & \cellcolor{gray!6}{-0.01 [-0.21, 0.18]} & \cellcolor{gray!6}{0.13 [-0.06, 0.33]} & \cellcolor{gray!6}{0.12 [-0.07, 0.31]}\\
Team 0\% / Opp. 25\% & 0.21 [0.00, 0.41] & -0.29 [-0.50, -0.09] & 0.33 [0.12, 0.54] & 0.27 [0.06, 0.49] & 0.46 [0.24, 0.67] & 0.25 [0.04, 0.47] & 0.31 [0.10, 0.53] & 0.54 [0.32, 0.76]\\
\cellcolor{gray!6}{Team 0\% / Opp. 50\%} & \cellcolor{gray!6}{0.19 [-0.03, 0.40]} & \cellcolor{gray!6}{-0.02 [-0.25, 0.20]} & \cellcolor{gray!6}{0.71 [0.47, 0.94]} & \cellcolor{gray!6}{0.56 [0.33, 0.79]} & \cellcolor{gray!6}{0.78 [0.54, 1.01]} & \cellcolor{gray!6}{0.49 [0.26, 0.73]} & \cellcolor{gray!6}{0.51 [0.27, 0.74]} & \cellcolor{gray!6}{0.97 [0.73, 1.21]}\\
Team 0\% / Opp. 75\% & 0.07 [-0.16, 0.30] & 0.13 [-0.11, 0.36] & 1.10 [0.86, 1.35] & 0.77 [0.53, 1.01] & 1.04 [0.80, 1.28] & 0.75 [0.51, 0.99] & 0.70 [0.46, 0.94] & 1.37 [1.12, 1.62]\\
\cellcolor{gray!6}{Team 0\% / Opp. 100\%} & \cellcolor{gray!6}{-0.06 [-0.30, 0.19]} & \cellcolor{gray!6}{0.28 [0.03, 0.52]} & \cellcolor{gray!6}{1.49 [1.22, 1.75]} & \cellcolor{gray!6}{1.02 [0.76, 1.27]} & \cellcolor{gray!6}{1.31 [1.05, 1.57]} & \cellcolor{gray!6}{0.95 [0.69, 1.20]} & \cellcolor{gray!6}{0.75 [0.50, 1.01]} & \cellcolor{gray!6}{1.64 [1.38, 1.90]}\\
\addlinespace
Team 25\% / Opp. 0\% & -0.04 [-0.26, 0.18] & -0.77 [-0.98, -0.56] & -0.33 [-0.54, -0.12] & -0.27 [-0.48, -0.05] & -0.13 [-0.34, 0.09] & -0.28 [-0.49, -0.07] & -0.14 [-0.35, 0.07] & -0.12 [-0.33, 0.09]\\
\cellcolor{gray!6}{Team 25\% / Opp. 25\%} & \cellcolor{gray!6}{-0.09 [-0.31, 0.14]} & \cellcolor{gray!6}{-0.55 [-0.77, -0.32]} & \cellcolor{gray!6}{0.10 [-0.13, 0.33]} & \cellcolor{gray!6}{0.02 [-0.21, 0.25]} & \cellcolor{gray!6}{0.20 [-0.03, 0.43]} & \cellcolor{gray!6}{-0.01 [-0.24, 0.22]} & \cellcolor{gray!6}{0.03 [-0.19, 0.26]} & \cellcolor{gray!6}{0.30 [0.07, 0.53]}\\
Team 25\% / Opp. 50\% & -0.11 [-0.35, 0.13] & -0.28 [-0.52, -0.04] & 0.48 [0.22, 0.73] & 0.30 [0.05, 0.55] & 0.52 [0.27, 0.77] & 0.23 [-0.02, 0.48] & 0.23 [-0.02, 0.48] & 0.73 [0.47, 0.98]\\
\cellcolor{gray!6}{Team 25\% / Opp. 75\%} & \cellcolor{gray!6}{-0.22 [-0.47, 0.03]} & \cellcolor{gray!6}{-0.13 [-0.38, 0.12]} & \cellcolor{gray!6}{0.87 [0.61, 1.13]} & \cellcolor{gray!6}{0.51 [0.26, 0.77]} & \cellcolor{gray!6}{0.79 [0.53, 1.05]} & \cellcolor{gray!6}{0.49 [0.23, 0.74]} & \cellcolor{gray!6}{0.42 [0.17, 0.67]} & \cellcolor{gray!6}{1.13 [0.87, 1.39]}\\
Team 25\% / Opp. 100\% & -0.35 [-0.62, -0.09] & 0.02 [-0.24, 0.28] & 1.25 [0.98, 1.53] & 0.76 [0.49, 1.03] & 1.05 [0.78, 1.33] & 0.68 [0.41, 0.95] & 0.48 [0.21, 0.74] & 1.40 [1.12, 1.67]\\
\addlinespace
\cellcolor{gray!6}{Team 50\% / Opp. 0\%} & \cellcolor{gray!6}{-0.34 [-0.58, -0.10]} & \cellcolor{gray!6}{-1.04 [-1.27, -0.80]} & \cellcolor{gray!6}{-0.57 [-0.81, -0.34]} & \cellcolor{gray!6}{-0.52 [-0.75, -0.29]} & \cellcolor{gray!6}{-0.37 [-0.60, -0.14]} & \cellcolor{gray!6}{-0.54 [-0.77, -0.31]} & \cellcolor{gray!6}{-0.42 [-0.65, -0.19]} & \cellcolor{gray!6}{-0.37 [-0.60, -0.15]}\\
Team 50\% / Opp. 25\% & -0.38 [-0.63, -0.13] & -0.81 [-1.06, -0.57] & -0.15 [-0.40, 0.10] & -0.24 [-0.49, 0.01] & -0.05 [-0.30, 0.20] & -0.28 [-0.52, -0.03] & -0.24 [-0.49, 0.01] & 0.05 [-0.20, 0.29]\\
\cellcolor{gray!6}{Team 50\% / Opp. 50\%} & \cellcolor{gray!6}{-0.40 [-0.67, -0.14]} & \cellcolor{gray!6}{-0.54 [-0.81, -0.28]} & \cellcolor{gray!6}{0.23 [-0.04, 0.50]} & \cellcolor{gray!6}{0.05 [-0.22, 0.31]} & \cellcolor{gray!6}{0.27 [0.00, 0.54]} & \cellcolor{gray!6}{-0.03 [-0.30, 0.23]} & \cellcolor{gray!6}{-0.05 [-0.31, 0.22]} & \cellcolor{gray!6}{0.48 [0.21, 0.75]}\\
Team 50\% / Opp. 75\% & -0.52 [-0.78, -0.26] & -0.40 [-0.66, -0.13] & 0.63 [0.35, 0.91] & 0.26 [-0.01, 0.53] & 0.54 [0.27, 0.81] & 0.22 [-0.05, 0.49] & 0.15 [-0.12, 0.41] & 0.88 [0.60, 1.15]\\
\cellcolor{gray!6}{Team 50\% / Opp. 100\%} & \cellcolor{gray!6}{-0.65 [-0.93, -0.37]} & \cellcolor{gray!6}{-0.24 [-0.52, 0.03]} & \cellcolor{gray!6}{1.01 [0.72, 1.30]} & \cellcolor{gray!6}{0.51 [0.22, 0.79]} & \cellcolor{gray!6}{0.81 [0.52, 1.10]} & \cellcolor{gray!6}{0.42 [0.13, 0.71]} & \cellcolor{gray!6}{0.20 [-0.08, 0.48]} & \cellcolor{gray!6}{1.15 [0.86, 1.44]}\\
\addlinespace
Team 75\% / Opp. 0\% & -0.62 [-0.86, -0.38] & -1.26 [-1.50, -1.03] & -0.77 [-1.01, -0.53] & -0.75 [-0.99, -0.50] & -0.59 [-0.83, -0.35] & -0.75 [-0.99, -0.51] & -0.63 [-0.87, -0.39] & -0.55 [-0.79, -0.31]\\
\cellcolor{gray!6}{Team 75\% / Opp. 25\%} & \cellcolor{gray!6}{-0.66 [-0.92, -0.41]} & \cellcolor{gray!6}{-1.04 [-1.29, -0.79]} & \cellcolor{gray!6}{-0.34 [-0.60, -0.09]} & \cellcolor{gray!6}{-0.46 [-0.72, -0.21]} & \cellcolor{gray!6}{-0.27 [-0.52, -0.01]} & \cellcolor{gray!6}{-0.49 [-0.75, -0.23]} & \cellcolor{gray!6}{-0.46 [-0.71, -0.20]} & \cellcolor{gray!6}{-0.13 [-0.39, 0.13]}\\
Team 75\% / Opp. 50\% & -0.69 [-0.95, -0.42] & -0.77 [-1.04, -0.51] & 0.04 [-0.24, 0.31] & -0.18 [-0.45, 0.09] & 0.05 [-0.22, 0.33] & -0.25 [-0.52, 0.03] & -0.26 [-0.54, 0.01] & 0.30 [0.02, 0.58]\\
\cellcolor{gray!6}{Team 75\% / Opp. 75\%} & \cellcolor{gray!6}{-0.80 [-1.07, -0.53]} & \cellcolor{gray!6}{-0.63 [-0.90, -0.35]} & \cellcolor{gray!6}{0.43 [0.15, 0.72]} & \cellcolor{gray!6}{0.03 [-0.25, 0.31]} & \cellcolor{gray!6}{0.32 [0.04, 0.60]} & \cellcolor{gray!6}{0.01 [-0.27, 0.29]} & \cellcolor{gray!6}{-0.07 [-0.35, 0.21]} & \cellcolor{gray!6}{0.70 [0.41, 0.99]}\\
Team 75\% / Opp. 100\% & -0.93 [-1.21, -0.65] & -0.47 [-0.75, -0.19] & 0.81 [0.52, 1.11] & 0.28 [-0.01, 0.57] & 0.59 [0.29, 0.88] & 0.21 [-0.09, 0.50] & -0.01 [-0.30, 0.28] & 0.97 [0.67, 1.27]\\
\addlinespace
\cellcolor{gray!6}{Team 100\% / Opp. 0\%} & \cellcolor{gray!6}{-0.80 [-1.06, -0.53]} & \cellcolor{gray!6}{-1.41 [-1.67, -1.16]} & \cellcolor{gray!6}{-0.90 [-1.15, -0.64]} & \cellcolor{gray!6}{-0.91 [-1.16, -0.65]} & \cellcolor{gray!6}{-0.77 [-1.03, -0.51]} & \cellcolor{gray!6}{-0.96 [-1.21, -0.70]} & \cellcolor{gray!6}{-0.87 [-1.12, -0.61]} & \cellcolor{gray!6}{-0.77 [-1.02, -0.51]}\\
Team 100\% / Opp. 25\% & -0.84 [-1.11, -0.57] & -1.19 [-1.46, -0.92] & -0.47 [-0.74, -0.20] & -0.62 [-0.90, -0.35] & -0.44 [-0.71, -0.17] & -0.69 [-0.97, -0.42] & -0.69 [-0.96, -0.42] & -0.35 [-0.62, -0.07]\\
\cellcolor{gray!6}{Team 100\% / Opp. 50\%} & \cellcolor{gray!6}{-0.86 [-1.14, -0.58]} & \cellcolor{gray!6}{-0.92 [-1.20, -0.64]} & \cellcolor{gray!6}{-0.09 [-0.38, 0.20]} & \cellcolor{gray!6}{-0.34 [-0.63, -0.05]} & \cellcolor{gray!6}{-0.12 [-0.41, 0.17]} & \cellcolor{gray!6}{-0.45 [-0.74, -0.16]} & \cellcolor{gray!6}{-0.50 [-0.78, -0.21]} & \cellcolor{gray!6}{0.08 [-0.21, 0.38]}\\
Team 100\% / Opp. 75\% & -0.98 [-1.27, -0.69] & -0.77 [-1.06, -0.48] & 0.30 [0.01, 0.60] & -0.13 [-0.42, 0.17] & 0.14 [-0.15, 0.44] & -0.19 [-0.48, 0.10] & -0.30 [-0.59, -0.01] & 0.48 [0.19, 0.78]\\
\cellcolor{gray!6}{Team 100\% / Opp. 100\%} & \cellcolor{gray!6}{-1.11 [-1.41, -0.81]} & \cellcolor{gray!6}{-0.62 [-0.92, -0.32]} & \cellcolor{gray!6}{0.69 [0.38, 1.00]} & \cellcolor{gray!6}{0.12 [-0.19, 0.43]} & \cellcolor{gray!6}{0.41 [0.10, 0.72]} & \cellcolor{gray!6}{0.00 [-0.30, 0.31]} & \cellcolor{gray!6}{-0.25 [-0.55, 0.06]} & \cellcolor{gray!6}{0.75 [0.44, 1.07]}\\
\bottomrule
\end{tabular}}
\end{table}
\end{landscape}

\hypertarget{discussion}{%
\section{Discussion}\label{discussion}}

\ldots{}

Our analysis of shot statistics from the first season of the Super Shot
demonstrated an approximate four-times higher likelihood of missing from
the outer versus inner circle, irrespective of the match period
(i.e.~standard vs.~Super Shot period). While the 95\% CIs did overlap
between the two periods, a slightly higher risk of missing from the
outer versus inner circle was present in the Super Shot period. Our
present findings conflict with our earlier work from the 2018 Super
Netball season where we observed an approximate two-times higher risk of
missing from the outer versus inner circle --- and suggested that the
2:1 point ratio of the Super Shot represented `good' value relative to
the elevated risk of missing ({\textbf{???}}). Our present analysis
perhaps suggests this is no longer the case --- as the risk of missing
from the outer circle has approximately doubled, and now outweighs the
additional point value on offer. Across the different teams and periods,
We found in all but one instance that the risk of missing from the outer
versus inner circle was greater than two-times that of the inner circle.
The only instance where the 95\% CIs overlapped the 2:1 ratio was for
the Firebirds during the standard period. This specific case is
irrelevant, however, as the additional points were not available during
this time. The obvious difference between the present study and our
previous work ({\textbf{???}}) is the actual presence of the Super Shot
being on offer during the 2020 season. The elevated risk of missing from
the outer versus inner circle from the 2018 to 2020 Super Netball season
is similar to what we observed in originally comparing the 2018 Super
Netball to Fast5 (i.e.~where long range shots were also rewarded with
additional points) ({\textbf{???}}). Therefore, it seems apparent that
adding value to long-range shots in netball induces an adjustment in the
way either attacking and/or defensive teams play that elevates the risk
of missing long- versus close-range shots. There are a number of factors
that could be contributing to this elevated risk. The most likely reason
is an additional emphasis from defensive players on preventing or
contesting long-range shots given their added value. This may also have
an inverse effect on the difficulty of generating and taking standard
shot opportunities, where by virtue of defensive strategies focusing on
Super Shots could make it simpler for teams to get easier shots closer
to the post. The presence of the Super Shot may also introduce
psychological pressure on the shooting player and influence the chance
of success. A more thorough analysis of defensive strategies and
understanding shooting players perspectives around the Super Shot can
assist in understanding the mechanisms behind any elevated risks of
missing long-range shots with added rewards.

Despite the Super Shot potentially holding an unbalanced risk-reward
trade-off (in general), there are likely scenarios where it is an
attractive option or appropriate risk. When trailing by a large margin
with minimal time remaining, the one point on offer for a standard shot
may present very little value to the trailing team. In this scenario,
the Super Shot potentially becomes the only or default option.
Conversely, the leading team would likely adopt a `safe' approach and
minimise their Super Shot attempts. Our analysis also considered overall
team shooting statistics. The relative risk of missing a Super Shot may
change for an individual (i.e.~specialist) long-range shooter. If a team
possesses such a player, emphasising Super Shots could represent a
relatively valuable opportunity. Similarly, there is some evidence to
support the `hot-hand' premise in shooting sports ({\textbf{???}}). A
team may benefit from preferentially feeding a shooter possessing this
characteristic for Super Shot attempts during a Power 5 period.

Restricting Super Shot attempts appears to be a safe, but likely
limiting strategy for scoring during Power 5 periods. Our simulations
examining potential scoring outputs with varying Super Shot proportions
demonstrates this premise. Employing a low proportion (i.e.~\textless{}
30\%) of Super Shots resulted in relatively low to moderate, but
consistent, scoring outputs. The high probability of standard shot
success is the likely driving factor behind the consistent scoring with
lower Super Shot proportions (i.e.~higher standard shot proportions).
Conversely, a high proportion (i.e.~\textgreater{} 80\%) of Super Shots
resulted in a much wider spectrum of scoring outputs from relatively low
through to high. Increasing the proportion of Super Shots taken
generated a progressive increase in the maximum score achievable
(i.e.~higher ceiling), but also coincided with a decrease in the minimum
score achievable (i.e.~decreased floor). Teams taking a high proportion
of Super Shots during Power 5 periods likely expose themselves to
volatile scoring outcomes --- effectively `living or dying by the sword'
that is the Super Shot. \textbf{\emph{Anything to wrap-up this point?}}

Remaining discussion points\ldots{}

Our approach in the present paper to simulate Super Shot scoring periods
differs to our original work ({\textbf{???}}). Previously, we allocated
an overall success rate to shots from the inner versus outer circle
(i.e.~if the sampled probability was 50\%, a total of 50\% of shots were
counted as successful). This contrasts to our present work, where we
sampled and applied the probability of shot success to simulated
individual shots (i.e.~if the sampled probability is 50\%, the
individual shot being simulated has a 50\% probability of success). This
approach likely reflects an improvement on our analysis, better
representing the independent nature of shots in a netball match.

\begin{itemize}
\tightlist
\item
  Team specific values, any obvious differences? For example, one team
  may have had better success and therefore using a higher proportion in
  general may have led to higher percentage of `won' periods
\item
  Team vs.~team specific values and if they are different across various
  opponents. For example, better shooting success with high super shot
  proportions vs.~one team but not another? This may be more relevant if
  we use opponent specific probabilities of super shot success.
  Important that the lack of `defensive' presence within simulations is
  acknowledged as a limitation, in that we applied the same super shot
  probability rates for each team from their entire season, rather than
  individually vs.~their oposition team. Given we might have some
  relative risk of missing against different defensive opponents, this
  may actually reveal that this should be a consideration if one team is
  more effective with their defense
\item
  Practical considerations of work include strategising around super
  shot, with respect to how many to take perhaps depending on margin
  along with opposition, as well as own teams success in this realm
\end{itemize}

Discussion notes\ldots check paragraphs here

Our simulation data does, however, demonstrate potential value in using
the Super shot for certain teams and in certain scenarios. Times where
higher proportion of super shot was valuable? We incorporated variable
shot opportunity numbers based on league data, and hence the number of
shot opportunities varied across individual simulations for teams. This
was balanced, in that when teams received more shot opportunities, their
opposition received fewer. Across all simulations, the wining team
received more shots on x\% of times. This factor became more/less
evident across scenarios where a team took a higher proportion of super
shots, whereby the winning team had more shots in x\% of these
simulations. This firstly suggests that generating more shot
opportunities than your opponent is obviously beneficial, but
potentially awards you more flexibility when considering taking a
greater proportion of super shots.

Similarly, teams who were better with the Super shot fared better in
simulations with greater proportions of super shots, and vice versa for
teams that are worse. For example, the fever lost x\% of simulations
when they went heavy on the Super shot, vs.~X team who won x\% of
simulations when using a high \% proportion of super shots. This is not
surprising as the fever had the highest risk of missing from the outer
vs inner circle, particularly during Super shot periods. These findings
likely demonstrate a need for teams to play to their shooting strengths.

Looking at the margin summaries (mean +/- 95\% CI's) from each teams
competitive simulations across all opponents, certain strategies
appeared more or less favourable across different teams. For example,
where the Fever extended above 50\% of their shots as Super Shots, it
was typical for them to score less than their opponent in the Super Shot
period; whereas when they used no Super Shots they typically outscored
their opponents. Other trends\ldots?

Nonetheless, the simulated margins within the Super Shot period,
irrrespective of the strategies used, were typically low - rarely
exceeding 1.5 to 2 points in a typical simulation. At most we suggest
that optimising the Super Shot proportions for a given team and opponent
may yield (on average) a 1 or 2 point gain each quarter. This may still
be beneficial for teams, as this could equate to 4 to 8 points across an
entire match. However, it is important to note that this is the average
+/- 95\% CIs, and hence it will not occur the same each time.

\hypertarget{conclusion}{%
\section{Conclusion}\label{conclusion}}

\ldots{}

Scenario specific use\ldots{}

\hypertarget{bibliography-styles}{%
\section{Bibliography styles}\label{bibliography-styles}}

There are various bibliography styles available. You can select the
style of your choice in the preamble of this document. These styles are
Elsevier styles based on standard styles like Harvard and Vancouver.
Please use BibTeX~to generate your bibliography and include DOIs
whenever available.

Here are two sample references: Feynman and Vernon Jr. (1963; Dirac,
1953).

\hypertarget{references}{%
\section*{References}\label{references}}
\addcontentsline{toc}{section}{References}

\hypertarget{refs}{}
\begin{cslreferences}
\leavevmode\hypertarget{ref-Dirac1953888}{}%
Dirac, P.A.M., 1953. The lorentz transformation and absolute time.
Physica 19, 888--896.
doi:\href{https://doi.org/10.1016/S0031-8914(53)80099-6}{10.1016/S0031-8914(53)80099-6}

\leavevmode\hypertarget{ref-Feynman1963118}{}%
Feynman, R.P., Vernon Jr., F.L., 1963. The theory of a general quantum
system interacting with a linear dissipative system. Annals of Physics
24, 118--173.
doi:\href{https://doi.org/10.1016/0003-4916(63)90068-X}{10.1016/0003-4916(63)90068-X}
\end{cslreferences}


\end{document}

